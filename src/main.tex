\documentclass[11p]{article}
% Packages
\usepackage{amsmath}
\usepackage{graphicx}
\usepackage[swedish]{babel}
\usepackage[
    backend=biber,
    style=authoryear-ibid,
    sorting=ynt
]{biblatex}
\usepackage[utf8]{inputenc}
\usepackage[T1]{fontenc}
%Källor
\addbibresource{mall.bib}
\graphicspath{ {./images/} }

\title{PMmall \\ \small Fysik 1}
\author{Magnus Silverdal }
\date{\today}

\begin{document}
    \begin{titlepage}
        \begin{center}
            \vspace*{1cm}

            \Huge
            \textbf{Energiförsörjning}

            \vspace{0.5cm}
            \LARGE
            Kärnkraft
            \vspace{1.5cm}

            \textbf{Nova Lindberg}

            \vfill

            Ett PM om energiförsörjning \\
            Fysik 1

            \vspace{0.8cm}

            %\includegraphics[width=0.4\textwidth]{../images/NTI Gymnasiet_Symbol_print_svart.png}

            \Large
            Teknikprogrammet\\
            NTI Gymnasiet\\
            Umeå\\
            \today

        \end{center}
    \end{titlepage}
% Om arbetet är långt har det en innehållsförteckning, annars kan den utelämnas
    \tableofcontents
    \newpage
    \section{Hur fungerar ett kärnkraftverk?}
    I Naturskyddsförenigens artikel \parencite{naturskyddsföreningen} så förklara dom att det fungerar som en jättestor ångmaskin – men man skapar ånga genom att klyva atomer och driva turbin/generatorn som alstrar elektricitet med den ångan. De reaktormodeller som finns i Sverige idag fungerar genom att man kokar ”vanligt” vatten (lättvatten) i reaktorn och leder det upphettade vattnet eller ångan( beroende på vilken reaktormodell) till antingen en turbin eller en ånggenerator.

    I ett kärnkraftverk används kärnbränsle. Genom att klyva uranatomer i bränslet med hjälp av neutroner frigörs stora mängder energi som värmer vattnet. Vid uranklyvningen frigörs i sin tur nya neutroner. För att neutronerna ska fortsätta att klyva uranatomer krävs att de bromsas upp, modereras.

    \section{Vilken miljöpåverkan har ett kärnkraftverk globalt och lokalt?}
    I vattensfalls artikel \parencite{vattenfall}nämner dom att det är den kraftkällan som har minst KLIMAT påverkan och lägst koldioxidutsläpp sett under sin livslängd enligt de standardiserade testerna som finns. Den största påverkan är under konstruktion och brytning av uran  och cement.


    \section{Hur påverkar kärnkraften samhället \\ (ekonomi/politik/Konflikter/m.m.) lokalt och globalt?}
    Under byggtiden skapas många arbetstillfällen för att sedan vara en stor lokal arbetsgivare under de närmsta 40 åren(standard drifttid för ett kärnkraftverk),dock måste staten och politiker ta ansvar enligt de regelverk som finns och det ska finnas en oberoende myndighet som ska kontrollera att driften och avfallet kontrolleras på ett säkert sätt. Vid väpnande konflikter är säkerheten väldigt viktig, men de reaktormodellerna som finns i Sverige kan rent tekniskt inte drabbas av samma sorts härdsmälta/explosion som Tjernobyl.
    Men vid krig kan ett kärnkraftverk användas som ett hot av en anfallande makt (se \parencite{Zaporizjzija}Zaporizjzija i Ukraina nu)

    \printbibliography

\end{document}
