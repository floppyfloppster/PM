\documentclass[11p]{article}
% Packages
\usepackage{amsmath}
\usepackage{graphicx}
\usepackage[swedish]{babel}
\usepackage[
    backend=biber,
    style=authoryear-ibid,
    sorting=ynt
]{biblatex}
\usepackage[utf8]{inputenc}
\usepackage[T1]{fontenc}
%Källor
\addbibresource{mall.bib}
\graphicspath{ {./images/} }

\title{PMmall \\ \small Fysik 1}
\author{Magnus Silverdal }
\date{\today}

\begin{document}
    Hur fungerar ett kärnkraftverk?
    I Naturskyddsförenigens artikel så förklara dom att det fungerar som en jättestor ångmaskin – men man skapar ånga genom att klyva atomer och driva turbin/generatorn som alstrar elektricitet med den ångan. De reaktormodeller som finns i Sverige idag fungerar genom att man kokar ”vanligt” vatten (lättvatten) i reaktorn och leder det upphettade vattnet eller ångan( beroende på vilken reaktormodell) till antingen en turbin eller en ånggenerator.

    I ett kärnkraftverk används kärnbränsle. Genom att klyva uranatomer i bränslet med hjälp av neutroner frigörs stora mängder energi som värmer vattnet. Vid uranklyvningen frigörs i sin tur nya neutroner. För att neutronerna ska fortsätta att klyva uranatomer krävs att de bromsas upp, modereras.

    Vilken miljöpåverkan har ett kärnkraftverk globalt och lokalt?
    Det är den kraftkällan som har minst KLIMAT påverkan och lägst koldioxidutsläpp sett under sin livslängd enligt de standardiserade testerna som finns. Den största påverkan är under konstruktion och brytning av uran  och cement.

    Och sett till närmiljöns påverkan så är den yta som ett kärnkraftverk tar upp minst, speciellt då man ser till den effekt som levereras till elnätet.

    Avfallsfrågan är också en fråga som har fått en lokal miljöpolitisk fråga. Att skapa plats för ett förvar för det kärnavfall som oundvikligen bildas under drift och avveckling är oftast en lång process och kräver en långsiktigt förvaringslösning.


    Hur påverkar kärnkraften samhället (ekonomi/politik/Konflikter/m.m.) lokalt och globalt?
    Under byggtiden skapas många arbetstillfällen för att sedan vara en stor lokal arbetsgivare under de närmsta 40 åren(standard drifttid för ett kärnkraftverk), dock måste staten ta ansvar enligt de regelverk som finns (IAEA) och det ska finnas en oberoende myndighet som ska kontrollera att driften och avfallet kontrolleras på ett säkert sätt. (Sen är politiska beslut så som i Sverige sådant som kan ställa till det för ägarna av kärnkraftsverken, tvingad avveckling/stängning m.m.) Vid väpnande konflikter är säkerheten väldigt viktig, men de reaktormodellerna som finns i Sverige kan rent tekniskt inte drabbas av samma sorts härdsmälta/explosion som Tjernobyl. Intressant att notera är att där kärnkraftverken byggs så finns det under driften ofta ett väldigt starkt stöd bland befolkning (Ringhals, Forsmark t.ex.)

    Men vid krig kan ett kärnkraftverk användas som ett hot av en anfallande makt (se Zaporizjzija i Ukraina nu)
    \begin{titlepage}
        \begin{center}
            \vspace*{1cm}

            \Huge
            \textbf{Title}

            \vspace{0.5cm}
            \LARGE
            Subtitle

            \vspace{1.5cm}

            \textbf{Author Name}

            \vfill

            Ett PM om energiförsörjning \\
            Fysik 1

            \vspace{0.8cm}

            \includegraphics[width=0.4\textwidth]{../images/NTI Gymnasiet_Symbol_print_svart.png}

            \Large
            Teknikprogrammet\\
            NTI Gymnasiet\\
            Umeå\\
            \today

        \end{center}
    \end{titlepage}
% Om arbetet är långt har det en innehållsförteckning, annars kan den utelämnas
    \tableofcontents
    \newpage
    
    \section{Disposition hos ett PM}
    Ett PM har den mest informella strukturen av de vetenskapliga texterna. Det är egentligen bara en sammanställning av kunskap men för att den ska bli lite lättare att ta sig an brukar det finnas en inledning där syfte och frågeställningar redovisas och en avslutning där du kan dra slutsatser. Rubrikerna kan döpas valfritt, speciellt de som finns i huvuddelen av texten beror på vad den handlar om. Se nedan för ett exempel.

    \section{Inledning}
    Beskriv varför detta ämne är intressant eller viktigt. Vad är syftet med texten?
    \subsection{frågeställningar}
    rada upp dina frågor i punktform
    \begin{enumerate}
        \item Fråga 1?
        \item Fråga 2?
        \item Fråga 3?
    \end{enumerate}

    \section{Resultat}
    Här kommer allt med massor av mer rubriker och underrubriker
    \subsection{Vindkraft, så fungerar det}
    \subsection{Globala miljökonsekvenser av kärnkraft}
    \subsection{Lokal miljöpåverkan av ett vattenkraftverk}
    \subsection{Solkraft bidrar till att minska konflikter om oljetillgångar i världen}
    \subsection{}

    \section{Slutsatser}
    Här kan du dra slutsatser eller sammanfatta ditt resultat

% Mer saker som du kan ha nytta av.

    \section{Referenser}
    Referenser i text kan skrivas på två sätt: Enligt \textcite{Jens} kan man använde två typer av referenser, inbäddade i texten eller efter ett fakta \parencite{Fraenkel}. Ett till test för att se hur det ser ut \parencite[sid 55]{fermi}.

    \section{Annat som kan vara bra att veta}
    Om du vill ha kodstil och få med alla tecken kan du använda verbatim. då kan du skriva \verb|abcd!"#| utan problem...

    Citat skrivs mellan de konstiga symbolerna \verb|``| och \verb|''| för att de ska se bra ut ``se bra ut!''.
    \subsection{En underrubrik}
    \subsubsection{En underunderrubrik}
    \subsection{Ekvationer}
    Det är lätt att skriva matematik i \LaTeX

    \begin{equation}
        F = G \frac{M m}{r^2}
        \label{grav}
    \end{equation}

    Ekvation (\ref{grav}) känner ni igen...

    \subsection{figurer}
    Bilder placeras enklast på detta sätt. placeringen bestämmer \LaTeX och vi kan bara föreslå (h)är, (t)opp eller (b)otten. Ett utropstecken före tvingar lite mer men inte absolut. I bild \ref{varg} visas en varg
    \begin{figure}[!h]
        \includegraphics[width=0.8\textwidth]{../images/accelerationTime.png}
        \caption{Acceleration-tid diagram. Källa: Impuls Fysik 1}
        \label{varg}
    \end{figure}
    \printbibliography

\end{document}
